\documentclass[11pt]{article}

\usepackage[english]{babel}
\usepackage{times}
\usepackage[cm]{fullpage}
\usepackage{amsmath}
\usepackage{mathtools}
\usepackage{amssymb}
\usepackage{geometry}
\usepackage{graphicx}
\usepackage{caption}
\usepackage{float}
\usepackage{makecell}
\usepackage{xspace}
\usepackage{color}
\usepackage{etoolbox}
\usepackage[]{longtable}

\addto{\captionsenglish}{\renewcommand{\abstractname}{\vspace{-2\baselineskip}}}

%%%%%% FONTS %%%%%%
\usepackage{fontspec}
\usepackage{unicode-math}

\setmainfont{STIXGeneral}[
  Extension = .otf ,
  Path = stix/ ,
  UprightFont = *-Regular ,
  BoldFont = *-Bold ,
  ItalicFont = *-Italic ,
  BoldItalicFont = *-BoldItalic ,
]
\setsansfont{calibri}[
  Extension = .ttf ,
  Path = calibri/ ,
  BoldFont = *b ,
  ItalicFont = *i ,
  BoldItalicFont = *z ,
]
\setmonofont{DroidSansMono}[
  Extension = .ttf ,
  Path = droid/ ,
  Scale=MatchLowercase ,
]
\setmathfont{STIX Math}

%%%%%% COMMANDS %%%%%%

\newcommand\todo[1]{\textcolor{red}{TODO: #1}}

\newcommand{\varstyle}[1]{\notblank{#1}{\textsf{$<$#1$>$}}{}}
\newcommand{\rul}[2]{\varstyle{#1} $\rightarrow$ #2}

\newcommand\imp{\textsc{Imp}\xspace}
\newcommand\inttt{\texttt{int32}\xspace}

\title{Introduction to language theory and compiling \\ Project -- Part 3}
\author{Antoine Passemiers -- Alexis Reynouard}

\pagestyle{plain}
\setcounter{tocdepth}{2}

\setlength{\parindent}{0em}
\setlength{\parskip}{1em plus0.5em minus0.3em}

\begin{document}
\pagenumbering{Roman}
\maketitle
\thispagestyle{empty}

\vfill
\begin{abstract}
In this work, we implemented an actual compiler for the \imp language.

Given an unambiguous LL(1) grammar (see previous report on how to produce it and how the program may help)
and a source file, the compiler produces an LLVM-assambly language (LLVM IR) .ll file which may be compiled to LLVM
bitcode with the LLVM assembler \textsf{llvm-as(1)}.
So we extended the previously given \imp grammar to handle some new features
and made improvements to the lexer and parser.

We will begin with some choices we did to implement the compiler.
Second we will introduces the new additional features of the language.
Finally, we will present the new enriched \imp grammar,
then the actual code generator and the additions to the implementation.
\end{abstract}

\vfill
\tableofcontents

\vfill
\clearpage
\setcounter{page}{1}
\pagenumbering{arabic}

\section{Assumptions, limitations and implementation choices}

First it worth noting that \imp does not know about scopes.
All the variables are global. \todo{and for functions arguments?}

The \imp variables are translated into named LLVM IR variables.
The temporary variables needed are implemented as unnamed LLVM IR variables.

The \imp language only support the \inttt type.
Any expressions which have a value have a \inttt value.
Notably, all functions which does not explicitly return a value return 0.

\section{Bonus features}

\begin{itemize}
\item Fonctions (definition et appel, arite au choix)
\item Mot cles rand, import, function
\item Gestion de librairie standard
\item Gestion d'exceptions
\item Optimization (faudrait qu'on checke les flags de llvm pour qu'il optimise les trucs immondes qu'on genere)
\item Si tu trouves des trucs facile a ajouter, ne te prives pas
\end{itemize}

\section{Augmenting \imp's syntax and grammar}

\subsection{Regular expressions}

The lexer was extended to read the new keywords:
\texttt{rand}, \texttt{function}, \texttt{return}, \texttt{import}, \texttt{,} (comma),
as well as two new ``kinds of identifier'': \textsf{FuncName} and \textsf{ModuleName}
which are respectively identifiers for functions (begins with a ``@'')
and identifiers for modules (begins with a ``\_'').

\begin{center}{\sffamily
\begin{tabular}{l @{ = } l}
  Identifier     & \{Alpha\}\{AlphaNumeric\}* \\
  FuncName       & @\{Identifier\} \\
  ModuleName     & \_\{Identifier\} \\
\end{tabular}
}\end{center}

\subsection{Grammar rules}

Here is the new grammar for the \imp language.
The main additions and changes are marked with an asterisk (\textsf{*}).

One can note tree main changes:
\begin{itemize}
  \item \varstyle{ParamList} was introduced to define functions (with \varstyle{Define}),
  \item \varstyle{ArgList} was introduced to call functions (with \varstyle{Call}),
  \item \varstyle{Assign} was modified to allow to assign a variable with the result of a function call.
\end{itemize}


\newcounter{Rule}
\newsavebox{\varbox}
\begin{longtable}{
    >{}r<{\sffamily[\stepcounter{Rule}\theRule]}
    l
    @{ $\rightarrow$ } >{\ttfamily}l<{\ttfamily}
  }
  & \varstyle{Program} & begin \varstyle{Code} end \\
  & \varstyle{Code} & \varstyle{InstList} \\
  &  & epsilon \\
  & \varstyle{Instruction} & \varstyle{Assign} \\
  &  & \varstyle{If} \\
  &  & \varstyle{While} \\
  &  & \varstyle{For} \\
  &  & \varstyle{Print} \\
  &  & \varstyle{Read} \\
* &  & \varstyle{Rand} \\
* &  & \varstyle{Define} \\
* &  & \varstyle{Return} \\
* &  & \varstyle{Call} \\
* &  & \varstyle{Import} \\
* & \varstyle{Define} & function [FuncName] ( \varstyle{ParamList} ) do \varstyle{Code} end \\
* & \varstyle{Return} & return \varstyle{ExprArith-p0} \\
* & \varstyle{Import} & import [ModuleName] \\
* & \varstyle{Call} & [FuncName] ( \varstyle{ArgList} ) \\
* & \varstyle{ArgList} & epsilon \\
* & \varstyle{ArgList} & \varstyle{ExprArith-p0} \varstyle{ArgList-Tail} \\
* & \varstyle{ArgList-Tail} & epsilon \\
* &  & , \varstyle{ArgList} \\
* & \varstyle{ParamList} & epsilon \\
* & \varstyle{ParamList} & [VarName] \varstyle{ParamList-Tail} \\
* & \varstyle{ParamList-Tail} & epsilon \\
* &  & , \varstyle{ParamList} \\
* & \varstyle{Assign} & [VarName] := \varstyle{Assign-Tail} \\
* & \varstyle{Assign-Tail} & \varstyle{Call} \\
* &  & \varstyle{ExprArith-p0} \\
  & \varstyle{Atom} & [VarName] \\
  &  & [Number] \\
  &  & ( \varstyle{ExprArith-p0} ) \\
  &  & - \varstyle{Atom} \\
  & \varstyle{Op-p0} & + \\
  &  & - \\
  & \varstyle{Op-p1} & * \\
  &  & / \\
  & \varstyle{Cond-p0-j} & or \varstyle{Cond-p1} \\
  & \varstyle{Cond-p0-i} & \varstyle{Cond-p1} \\
  & \varstyle{Cond-p1-j} & and \varstyle{Cond-p2} \\
  & \varstyle{Cond-p1-i} & \varstyle{Cond-p2} \\
  & \varstyle{Cond-p2} & not \varstyle{SimpleCond} \\
  &  & \varstyle{SimpleCond} \\
  & \varstyle{SimpleCond} & \varstyle{ExprArith-p0} \varstyle{Comp} \varstyle{ExprArith-p0} \\
  & \varstyle{Comp} & = \\
  &  & >= \\
  &  & > \\
  &  & <= \\
  &  & < \\
  &  & <> \\
  & \varstyle{While} & while \varstyle{Cond-p0} do \varstyle{Code} done \\
  & \varstyle{Print} & print ( [VarName] ) \\
  & \varstyle{Read} & read ( [VarName] ) \\
* & \varstyle{Rand} & rand ( [VarName] ) \\
  & \varstyle{InstList} & \varstyle{Instruction} \varstyle{InstList-Tail} \\
  & \varstyle{InstList-Tail} & ; \varstyle{InstList} \\
  &  & epsilon \\
  & \varstyle{If} & if \varstyle{Cond-p0} then \varstyle{Code} \varstyle{If-Tail} \\
  & \varstyle{If-Tail} & endif \\
  &  & else \varstyle{Code} endif \\
  & \varstyle{For} & for [VarName] from \varstyle{ExprArith-p0} \varstyle{For-Tail} \\
  & \varstyle{For-Tail} & to \varstyle{ExprArith-p0} do \varstyle{Code} done \\
  &  & by \varstyle{ExprArith-p0} to \varstyle{ExprArith-p0} do \varstyle{Code} done \\
  & \varstyle{ExprArith-p0-j} & \varstyle{Op-p0} \varstyle{ExprArith-p1} \\
  & \varstyle{ExprArith-p0-i} & \varstyle{ExprArith-p1} \\
  & \varstyle{ExprArith-p1-j} & \varstyle{Op-p1} \varstyle{Atom} \\
  & \varstyle{ExprArith-p1-i} & \varstyle{Atom} \\
  & \varstyle{ExprArith-p0} & \varstyle{ExprArith-p0-i} \varstyle{ExprArith-p0-j} \\
  & \varstyle{ExprArith-p0-j} & epsilon \\
  & \varstyle{ExprArith-p1} & \varstyle{ExprArith-p1-i} \varstyle{ExprArith-p1-j} \\
  & \varstyle{ExprArith-p1-j} & epsilon \\
  & \varstyle{Cond-p0} & \varstyle{Cond-p0-i} \varstyle{Cond-p0-j} \\
  & \varstyle{Cond-p0-j} & epsilon \\
  & \varstyle{Cond-p1} & \varstyle{Cond-p1-i} \varstyle{Cond-p1-j} \\
  & \varstyle{Cond-p1-j} & epsilon \\
\end{longtable}

\section{Implementation}

\subsection{Improvements in both lexer and parser}

\subsubsection{Error handling}

 erreur de syntax, nom de module non existant, fonction non definie...

\subsection{LLVM code generator}

recursive descent code generator ?

On a hard code une methode par variable de la grammaire.
Quand c'est necessaire, une methode renvoie le nom d'une variable temporaire (non nommee)
qui stocke le resultat de l'expression.

Exemple: a := (7 + 9)

7 est stockée dans une variable non nommee, puis de meme pour 9, puis 7 + 9 encore dans une autre, puis cette derniere est stockee
dans la variable \%a avec un store.

\end{document}

% vim: set spell spelllang=en :
