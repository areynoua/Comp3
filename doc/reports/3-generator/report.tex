\documentclass[12pt]{report}

\usepackage[english]{babel}
\usepackage{times}
\usepackage[cm]{fullpage}
\usepackage[bottom]{footmisc}
\usepackage{chngcntr}
\usepackage{etoolbox}
\usepackage{amsmath}
\usepackage{mathtools}
\usepackage{geometry}
\usepackage{calrsfs}
\usepackage{graphicx}
\usepackage{caption}
\usepackage{float}
\usepackage{titlesec}
\usepackage{amssymb} 
\usepackage{makecell}
\usepackage{color}
\usepackage{enumerate}

\geometry{
 a4paper,
 total={170mm,257mm},
 left=20mm,
 top=20mm,
}

\newcommand\todo[1]{\textcolor{red}{TODO: #1}}

\titleformat{\chapter}{\normalfont\huge}{\thechapter.}{20pt}{\huge}


\title{Introduction to language theory and compiling \\ Project - Part 3}
\author{Antoine Passemiers \\ Alexis Reynouard}
\date{December 12, 2017}

\pagestyle{plain}

\setlength{\parindent}{0em}
\setlength{\parskip}{1em plus0.5em minus0.3em}

\begin{document}
\pagenumbering{Roman}
\maketitle
\setcounter{tocdepth}{3}
\setcounter{secnumdepth}{3}
\setcounter{chapter}{0}
\tableofcontents
\pagebreak
\clearpage
\setcounter{page}{1}
\pagenumbering{arabic}

\chapter{Assumptions}

\begin{itemize}
\item Pas de gestion du scope. Tout est variable global, meme pour les appels de fonction.
\item Les variables Imp sont nommés en llvm, toutes les variables temporaires sont non-nommees
\item Tout est int32 (meme les fonctions ne renvoyant rien renvoient i32 0)
\end{itemize}

\chapter{Augmenting Imp's syntax and grammar}

\section{Regular expressions}

On s'est simplifié la vie pour funcName, moduleName.
Mots clés faciles du genre import.

\section{Grammar rules}

arglist, paramlist... assign -> call

\chapter{Implementation}

\section{Improvements in both lexer and parser}

\subsection{Error handling}

 erreur de syntax, nom de module non existant, fonction non definie...

\section{LLVM code generator}

recursive descent code generator ?

On a hard code une methode par variable de la grammaire.
Quand c'est necessaire, une methode renvoie le nom d'une variable temporaire (non nommee)
qui stocke le resultat de l'expression.

Exemple: a := (7 + 9)

7 est stockée dans une variable non nommee, puis de meme pour 9, puis 7 + 9 encore dans une autre, puis cette derniere est stockee
dans la variable \%a avec un store.

\chapter{Bonus features}

\begin{itemize}
\item Fonctions (definition et appel, arite au choix)
\item Mot cles rand, import, function
\item Gestion de librairie standard
\item Gestion d'exceptions
\item Optimization (faudrait qu'on checke les flags de llvm pour qu'il optimise les trucs immondes qu'on genere)
\item Si tu trouves des trucs facile a ajouter, ne te prives pas
\end{itemize}

\end{document}
