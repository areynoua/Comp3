\documentclass[12pt]{report}

\usepackage[english]{babel}
\usepackage{times}
\usepackage[cm]{fullpage}
\usepackage[bottom]{footmisc}
\usepackage{chngcntr}
\usepackage{etoolbox}
\usepackage{amsmath}
\usepackage{mathtools}
\usepackage{geometry}
\usepackage{calrsfs}
\usepackage{graphicx}
\usepackage{caption}
\usepackage{float}
\usepackage{titlesec}
\usepackage{amssymb} 
\usepackage{tikz} 
\usepackage{tikz-qtree}
\usetikzlibrary{automata,positioning}

\tikzset{set/.style={draw,circle,inner sep=0pt,align=center}}

\geometry{
 a4paper,
 total={170mm,257mm},
 left=20mm,
 top=20mm,
}

\titleformat{\chapter}{\normalfont\huge}{\thechapter.}{20pt}{\huge}


\title{Introduction to language theory and compiling \\ Project - Part 1}
\author{Antoine Passemiers \\ Alexis Reynouard}
\date{October 8, 2017}

\pagestyle{plain}

\begin{document}
\pagenumbering{Roman}
\maketitle
\setcounter{tocdepth}{3}
\setcounter{secnumdepth}{3}
\setcounter{chapter}{0}
\tableofcontents
\pagebreak
\clearpage
\setcounter{page}{1}
\pagenumbering{arabic}


\chapter{Introduction}

\chapter{Bonus}

Lexing should be done with regular expressions, but nested comments can not be processed by regular expressions, because they don't have
a context-free grammar. Why ? I would say because of the pumping lemma. \\ \\
Two solutions: implement a recursive lexer (very complicated to compile), or use counters (Java can do that because it is Turing-complete).

\end{document}