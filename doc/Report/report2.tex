\documentclass[12pt]{report}

\usepackage[english]{babel}
\usepackage{times}
\usepackage{fullpage}
\usepackage[bottom]{footmisc}
\usepackage{chngcntr}
\usepackage{etoolbox}
\usepackage{amsmath}
\usepackage{mathtools}
\usepackage{geometry}
\usepackage{calrsfs}
\usepackage{graphicx}
\usepackage{caption}
\usepackage{float}
\usepackage{titlesec}
\usepackage{amssymb} 
\usepackage{makecell}
\usepackage{color}
\usepackage{enumerate}
\usepackage{array}

\newcommand\todo[1]{\textcolor{red}{TODO: #1}}

\titleformat{\chapter}{\normalfont\huge}{\thechapter.}{20pt}{\huge}

\title{Introduction to language theory and compiling \\ Project -- Part 2}
\author{Antoine Passemiers \\ Alexis Reynouard}

\pagestyle{plain}

\setlength{\parindent}{0em}
\setlength{\parskip}{1em plus0.5em minus0.3em}

\pagenumbering{Roman}
\setcounter{tocdepth}{3}
\setcounter{secnumdepth}{3}
\setcounter{chapter}{0}
\pagenumbering{arabic}

%FIXME: does not print <> if #1 empty
\newcommand{\varstyle}[1]{\notblank{#1}{\textsf{$<$#1$>$}}{}}

\begin{document}
\maketitle
\tableofcontents
\thispagestyle{empty}
\pagebreak
\setcounter{page}{1}
\clearpage

\chapter{Transforming Imp grammar}

This part of the project consists in implementing a $LL(k)$ parser for the Imp programming language. A $LL(k)$ parser is a recursive descent parser composed of:

\begin{itemize}
\item An input buffer, containing $k$ input tokens. Since we are considering a $LL(1)$ parser, the latter only considers one token at a time to decide how to 
grow the syntactic tree.
\item A stack containing the set of remaining terminals and non-terminals to process.
\item An action table, mapping the head 
\end{itemize}

\section{Removing useless rules}

\subsection{Unreachable variables}

\subsection{unproductive variables}

\section{Removing left-recursion and applying factorization}

\subsection{Left-recursion}

\subsection{Factorization}

\section{Removing ambiguity}

\subsection{Operator priority}

\subsection{Operator associativity}

\section{Resulting grammar}

\newcounter{Rule}
\newsavebox{\varbox}
\begin{tabular}{
    >{\sffamily[\stepcounter{Rule}\theRule}r<{]}
    >{\begin{lrbox}{\varbox}\sffamily}l<{\end{lrbox}\varstyle{\unhbox\varbox}}
    @{ $\rightarrow$ } >{\ttfamily}l<{\ttfamily}
  }
  & Program & begin \varstyle{Code} end \\
  & Code & $\epsilon$ \\
  & & \varstyle{InstList}
\end{tabular}
\end{document}
