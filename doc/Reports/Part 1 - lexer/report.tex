\documentclass[12pt]{report}

\usepackage[english]{babel}
\usepackage{times}
\usepackage[cm]{fullpage}
\usepackage[bottom]{footmisc}
\usepackage{chngcntr}
\usepackage{etoolbox}
\usepackage{amsmath}
\usepackage{mathtools}
\usepackage{geometry}
\usepackage{calrsfs}
\usepackage{graphicx}
\usepackage{caption}
\usepackage{float}
\usepackage{titlesec}
\usepackage{amssymb} 
\usepackage{makecell}
\usepackage{color}
\usepackage{enumerate}

\geometry{
 a4paper,
 total={170mm,257mm},
 left=20mm,
 top=20mm,
}

\newcommand\todo[1]{\textcolor{red}{TODO: #1}}

\titleformat{\chapter}{\normalfont\huge}{\thechapter.}{20pt}{\huge}


\title{Introduction to language theory and compiling \\ Project - Part 1}
\author{Antoine Passemiers \\ Alexis Reynouard}
\date{October 8, 2017}

\pagestyle{plain}

\setlength{\parindent}{0em}
\setlength{\parskip}{1em plus0.5em minus0.3em}

\begin{document}
\pagenumbering{Roman}
\maketitle
\setcounter{tocdepth}{3}
\setcounter{secnumdepth}{3}
\setcounter{chapter}{0}
\tableofcontents
\pagebreak
\clearpage
\setcounter{page}{1}
\pagenumbering{arabic}

\chapter{Assumptions}

This part of the project consists in designing and implementing a lexical analyzer for the Imp programming language,
from a given list of reserved keywords and a grammar for the language.

\begin{itemize}
  \item As suggested in the project brief, numbers are only made up of sequences of digits, thus can only
    be integers. Decimal numbers are not supposed to be recognized by the lexer. Negative integers are handled using
    the rule 16 from the Imp grammar: a negative number is an arithmetic expression composed of the unary operator
    ``\texttt{-}'' followed by a positive number. Here, the lexer must only be capable of tokenizing both the unary
    operator ``\texttt{-}'' and positive numbers, and thus not able to infer the existence of a negative number by itself. This
    will be implemented in the next parts of the project.
  \item Variable identifiers (varnames) are sequences of letters and digits, and necessarily start with a letter.  
    Therefore, they can not contain underscores. Also they are case sensitive.
  \item Since no lexical units have been provided for
    booleans nor for strings, the lexer does not handle them. In any case, booleans could be regarded as simple
    integers, and possibly constants could be defined in the Imp environment to support them. \\ For example:
    \texttt{TRUE := 1; FALSE := 0}
  \item We did not relax the assumption on token delimiters: two tokens may or may not be
    separated by a blank character. This allows tokens to be contiguous, \textit{i.e.} two tokens need to be separated
    by a blank character only if the first one ends with a letter and the last one starts with a letter.
    For instance, one can write:\\
    \texttt{begin read(a);read(b);while b<>0do c:=b;\\while a>=b do a:=a-b done;b:=a;a:=c done;print(a) end}
  \item Non-ASCII characters are not allowed, except in comment sections.
  \item Because the provided .out files do not contain EOS tokens, we decided not to insert them yet.
\end{itemize}

\chapter{Regular expressions}

\section{Notable regular expressions}

\begin{tabular}{|l >{\ttfamily}l p{13em}|} \hline
  \textbf{Name}
	& \textnormal{\textbf{Regular expression}}
	& \textbf{Description} \\ \hline
CommentBegin
	& \textbackslash(\textbackslash*
	& Opening comment symbol \\ \hline
CommentContent
	& (\textbackslash*[\textasciicircum \textbackslash)]$\vert$[\textasciicircum*])*\textbackslash*\textbackslash)
    & Well-formed comment: Match any sequence of characteres up to and including the first ``\texttt{*)}''. (Nested comments are forbidden.)  \\ \hline
Space
	& [ \textbackslash n\textbackslash r\textbackslash t\textbackslash f]*
	& Blank spaces: only space, new line,
vertical/horizontal tabulations are taken into account \\ \hline
AlphaUpperCase
	& [A-Z]
	& A single uppercase letter \\ \hline
AlphaLowerCase
	& [a-z]
	& A single lowercase letter \\ \hline
Alpha
	& \{AlphaUpperCase\}$\vert$\{AlphaLowerCase\}
	& A single alphabetical character \\ \hline
Digit
	& [0-9]
	& A single digit \\ \hline
AlphaNumeric
	& \{Alpha\}$\vert$\{Digit\}
	& A digit or an alphabetical character \\ \hline
Number
	& \{Digit\}+
	& A sequence of digits \\ \hline
Identifier
	& \{Alpha\}\{AlphaNumeric\}*
	& Sequence of characters that does not start with a digit \\ \hline
\end{tabular} \\ \\

\section{Trivial regular expressions}

\texttt{begin},
\texttt{end},
\texttt{;},
\texttt{:=},
\texttt{(},
\texttt{)},
\texttt{-},
\texttt{+},
\texttt{*},
\texttt{/},
\texttt{if},
\texttt{then},
\texttt{endif},
\texttt{else},
\texttt{not},
\texttt{and},
\texttt{or},
\texttt{=},
\texttt{>=},
\texttt{>},
\texttt{<=},
\texttt{<},
\texttt{<>},
\texttt{while},
\texttt{do},
\texttt{done},
\texttt{for},
\texttt{from},
\texttt{by},
\texttt{to},
\texttt{print},
\texttt{read},
\texttt{\textbackslash{}*\textbackslash)}.

The last regular expression matches ``end of comment'' tokens. This should never happen as this token is already
considered as part of the comment. This is to raise an exception when it occurs outside any comment. Another
option would be to consider two lexical units: \texttt{TIMES} and \texttt{RPAREN}. Since this sequence does not make sense
according to the Imp grammar, we chose to raise an exception yet in order to detect nested comment more quickly.

\chapter{Implementation}

\section{Lexical analyzer}

\subsection{Lexical unit matching}

Besides the JFlex-default state $YYINITIAL$, we defined a new state $COMMENT$ which is active when the opening comment symbol ($CommentBegin$) is encountered. The system goes back to $YYINITIAL$ state once a comment content is matched, according to the regex $CommentContent$.
All the other regular expressions can be matched only in the $YYINITIAL$ state, because comment sections can also contain reserved keywords, but the latter
must not be considered as such inside comments.


The main problem is that \textit{reserved keywords} can be matched with the varname regular expression as well. As a
result, more than one token may be recognized at once, and JFlex uses two rules of preference to decide which one to
return. JFlex first selects the longest match, and then chooses the first matched string among the longest matches.
Thus, a common solution to keep the keywords ``reserved'' is to give priority to the keyword regular expressions (all
but varname) by putting them ahead in the source code. The identifier regular expression has been added after all the others.


Except for comment contents, each new matched regular expression induces the instantiation of a new \textit{Symbol}
object.  The matched string is stored as the \textit{value} attribute, the \textit{line} attribute is set to the current
line number and the \textit{column} attribute is set to the current column number. The \textit{line} attribute is used
by the symbol table for handling varnames (this will be explained in the next section).


We defined two custom exceptions, \textit{BadTerminalException} and \textit{BadTerminalContextException}. The first one occurs when the current string
$yytext$ doesn't match any of our defined regular expressions. In other words, a \textit{BadTerminalException} is thrown when the regular expression
\texttt{[\textasciicircum]} (whose match anything but has the lowest priority) is matched. In addition, a
\textit{BadTerminalContextException} is thrown when a closing comment symbol is matched inside a non-comment section.
Because the lexer is not supposed to throw exceptions, we implemented a Main Java class that wraps the execution of the
JFlex-generated class (called \textit{LexicalAnalyzer.java}) and catches all the potential exceptions. When an exception
is catched, a dedicated message is simply displayed in the standard output instead.

\subsection{Symbol table management}

We used a \textit{LinkedHashMap} from the java standard library to store the different symbols. We chose this kind of data structure rather than
a simple \textit{HashMap} because it is more convenient to keep track of the order of inserted symbols. Indeed, symbols are supposed to be
displayed in order of appearance.


Each time a lexical unit is matched, its value is hashed and compared to the indexes present in the linked hashmap. The unit is added to the linked hashmap 
if and only if its hash value is not already present. At the end of the program, the symbol table is shown by simply iterating over the hashmap's keyset. 
The first column of the table contains tokens (unit names) and the second column contains, for each token, the line number where the token has
been encountered for the first time during the program execution. This number is obtained by evaluating the variable \textit{yyline} at the moment when
the current token is matched.

\section{Testing}

Nine test .imp files have been written, as well as nine corresponding .out files, to test all the features of our lexer.
In addition, a bash script has been written to automatically apply the following procedure:

\begin{itemize}
\item Generate the Java lexer class using JFlex
\item Compile the latter using the javac command
\item Generate the javadoc
\item Generate an executable jar from the .class files
\item For each of the test.imp files, run the lexer on the .imp and compare the output file with the corresponding predefined .out file.

  The comparisons stop when one output does not correspond to the expected output and the differences are displayed.

  The number and type of blank characters is ignored when comparing outputs, but not the presence or absence of a space.
\end{itemize}

Here are short descriptions of our nine pairs of test files:

\begin{enumerate}
\item Example files provided with the project brief. Checks that the given the .imp file, the lexer generates a .out that is identical to the provided one.
\item Same .imp file with an additional comment and without the character of end of line at the end of the file.
  The output file should remain the same.
\item Example .imp file provided with the project, but without any unnecessary blanks.
  The output file should remain the same.
\item Empty code: a \textit{begin} token followed by a \textit{end} token. The symbol table should be empty.
\item Small code with a non-ASCII comment. The lexer should \textbf{not} throw a \textit{BadTerminalException}.
\item Comments, plus a single "*)" symbol. As explained it should lead to a \textit{BadTerminalContextException}.
\item Tokens without separators between them. As explained before, "2a" should be tokenized as "2" followed by "a".
\item Similar to the example Imp code but using negative numbers.
\item Checks that non-ASCII characters are actually not supported by our lexer. A \textit{BadTerminalException} message is expected in the output file.
\end{enumerate}

\chapter{Nested comments -- Bonus}

\newcommand{\stac}{``\texttt{(*}'' }
\newcommand{\stoc}{``\texttt{*)}'' }

The role of the lexer is to split the whole .imp file into a list of tokens.
For simplicity, we can say that this is done by matching each part of the file against regular expressions.
This is to say that, for every file part, we ask whether it is a word that belongs to the language described by a regular language or not.
The corresponding regular language tells us of which type of token the word is.

We expect comments to be deleted by the lexer so that they are not visible to the parser.
Thus, comments are words of a regular language that can be described informally this way: they start with an
\stac, contain any character and end with \emph{the first} \stoc.
This is the rule we used in the implementation.
We need to ``stop'' the comment at the first \stoc to avoid considering all the characters
between the first \stac and the last \stoc as part of the comment.

Lexing should be done with regular expressions, but nested comments can not be processed by regular expressions, because
they do not belong to regular languages. More specifically, it is impossible to know at compile time the number of states
required to handle nested comments.  For any fixed number of states, one will always be able to provide a comment with a
level of nesting such that it cannot be recognized by the lexer.  This is due to the \textit{pumping lemma for regular
languages}. Another way to say this is: when the closing comment symbol \stoc is matched, a \textbf{finite} automaton would not be able to
remember the number of \stac symbols already matched. For any number of matched opening comment symbols, the current state
would be the same: this is the intuition behind the \textit{pigeonhole principle}. As a result, we need to keep track of
this number.

We can think of two solutions: implement a recursive lexer (very complicated to compile),
or use counters (Java can do that because it is Turing-complete).
In fact these two solutions are two different point of views on the way to keep track of the ``comment level'' the lexer
is in. This is because the nested comments belong to the class of context-free languages (recognizable by pushdown
automata).
\begin{itemize}
  \item The first solution we propose is to use a stack and call a lex scanner recursively.
    Nested comments will still be recognized since they consist in both a left recursive grammar and a right recursive
    grammar.
    % The problem is that the stack size must be known at compile time, which makes the problem similar to the one
    % described earlier. Once again, we are not allowed to do so.
  \item %The second solution is more general and also requires a memory space that is theoretically infinite, since we are dealing with Turing machines.
    The second solution may be described by a procedure described as follows:
    \begin{itemize}
      \item{Initialize a counter $c$ to 0}
      \item{Check the current symbol. If it's a "(*", increment $c$. If it's a "*)", decrement $c$. Do nothing in other cases.}
      \item{Reject if $c$ becomes strictly negative.}
      \item{Match the next symbol.}
    \end{itemize}
\end{itemize}
JFlex already offers a fancy way to integrate custom Java methods in the automaton states. Our preference would be to
benefit from it and manipulate counters in the Java part of the code. This would be easier to design and to maintain.

Note: We are aware that it is preferable for efficiency purposes (when possible) to use automaton-based implementations
rather than calling Java code everywhere. Indeed JFlex is designed to optimize automata through minimization.  
\end{document}
