\section{First sets, follow sets, action table}

Because the objective is to build a LL(1) parser, which is a predictive parser, one does
need to design the predictive features of the parser. Given a variable on the stack, let's say
\varstyle{Code}, the parser is not supposed to know which rule to apply between the following two rules:

\begin{tabular}{lll}
  \varstyle{Program} & $\rightarrow$ & begin \varstyle{Code} end \\
  \varstyle{Code} & $\rightarrow$ & $\epsilon$ \\
  & $\rightarrow$ & \varstyle{InstList} \\
\end{tabular}

This is solved by looking at the next input token. If the latter is, let's say \textit{a}, the code will contain instructions.
As a result, rule \varstyle{Code} $\rightarrow$ \varstyle{InstList} should be applied.
However, if the next input token is \textit{end}, the code will be empty and rule \varstyle{Code} $\rightarrow \epsilon$
should be applied.

\subsection{First sets}

We will consider only look-aheads of one symbol at a time in the framework of this report.
The $First^1$ set of a symbol is the \textbf{set of first terminals} that can be derived from this symbol using
different rules where the left-hand part consists in this symbol. Let's adapt the definition of first sets
for LL(1) grammars:

\begin{equation}
  \begin{split}
    First^1(A) = \{ b \in T | A \Rightarrow_{Imp}^{*} bx \}
   \end{split}
\end{equation}

The first set of a terminal is the terminal itself. Indeed, with a terminal on top of the stack, there must necessarily be a \textbf{match}
with the next input token: the two terminals must be equals. As a result, the only first token that can be derived from the terminal is
the terminal itself.

This is more complex for variables. Let's take the previous example again. The first set of \varstyle{Code}
is composed both of \textit{end} token and the first set of \varstyle{InstList}. Because the second rule of the example
does not produce any terminal but $\epsilon$, the next symbol to be produced is the one that \textbf{follows in the parent production rule}.
This is the purpose of \textbf{follow sets}.
Regarding the third rule, \varstyle{InstList} is a variable (which does not produce a token by itself), so one has to explore its first set
before to be able to find first terminals.

The algorithm that computes first sets has been implemented as a \textbf{fixed-point iteration}: it starts by computing, for each terminal, 
its first set as the set containing only the terminal itself. Then it proceeds by reduction: for each rule producing sequences of symbols
from which first sets are known, it computes the first set of the left-hand variable.
The first set of a variable is then the \textbf{concatenation of first sets obtained} in all rules where this variable is in left-hand part, 
except for first sets containing $\epsilon$ (for the reason given earlier).

\subsection{Follow sets}

Because the right-hand part if a rule may produce $\epsilon$, the first set is \textbf{not sufficient} to predict which rule to use.
Let's consider rule $A \rightarrow w$, where the first set of w is the empty word. On that case, one can not infer the first terminal
to be produced after A has been pushed on the stack. Actually, this terminal comes after $A$. So we define the follow set of $A$ as
the set of terminals such that there exists a derivation from the root that gives $A a B$, where $a$ is a terminal.

Let's take our first example again. We get that $Follow^1(<Code>)$ is $\{$ \textit{end} $\}$ because the only rule where \varstyle{Code}
is on right-hand side contains a \textit{end} right after \varstyle{Code}. This implies that, when \varstyle{Code} $\rightarrow \epsilon$
is applied, the terminal that follows is \textit{end}. We must build the action table is such a way that the parser becomes able to predict
that rule \varstyle{Code} $\rightarrow \epsilon$ must be used, only by seeing \textit{end} as the next input token (look-ahead).

\subsection{$First^1$ and $Follow^1$ sets for Imp}

  %\resizebox{1.2\linewidth}{!}{%
    {\small\ttfamily
      \begin{longtable}{r p{7cm} p{7cm}}
\textnormal{Symbol} $A$ & $First^1(A)$ & $Follow^1(A)$\\ \hline
<If> & if  & else end endif ; done \\ \hline
<For> & for  & else end endif ; done \\ \hline
<Atom> & [VarName] ( - [Number]  & <= <> or ) * + do then - done / else and by end endif ; to < = > >= \\ \hline
<Code> & epsilon [VarName] print read for while if  & else end endif done \\ \hline
<Comp> & <= <> < = > >=  & [VarName] ( - [Number] \\ \hline
<Read> & read  & else end endif ; done \\ \hline
<Op-p0> & + -  & [VarName] ( - [Number] \\ \hline
<Op-p1> & * /  & [VarName] ( - [Number] \\ \hline
<Print> & print  & else end endif ; done \\ \hline
<While> & while  & else end endif ; done \\ \hline
<Assign> & [VarName]  & else end endif ; done \\ \hline
<Cond-p0> & [VarName] not ( - [Number]  & do then \\ \hline
<Cond-p1> & [VarName] not ( - [Number]  & or do then \\ \hline
<Cond-p2> & [VarName] not ( - [Number]  & or and do then \\ \hline
<If-Tail> & else endif  & else end endif ; done \\ \hline
<Program> & begin  & epsilon \\ \hline
<For-Tail> & by to  & else end endif ; done \\ \hline
<InstList> & [VarName] print read for while if  & else end endif done \\ \hline
<Cond-p0-i> & [VarName] not ( - [Number]  & or do then \\ \hline
<Cond-p0-j> & epsilon or  & do then \\ \hline
<Cond-p1-i> & [VarName] not ( - [Number]  & or and do then \\ \hline
<Cond-p1-j> & epsilon and  & or do then \\ \hline
<SimpleCond> & [VarName] ( - [Number]  & or and do then \\ \hline
<Instruction> & [VarName] print read for while if  & else end endif ; done \\ \hline
<ExprArith-p0> & [VarName] ( - [Number]  & <= <> or ) do then done else and by end endif ; to < = > >= \\ \hline
<ExprArith-p1> & [VarName] ( - [Number]  & <= <> or ) + do then - done else and by end endif ; to < = > >= \\ \hline
<InstList-Tail> & epsilon ;  & else end endif done \\ \hline
<ExprArith-p0-i> & [VarName] ( - [Number]  & <= <> or ) + do then - done else and by end endif ; to < = > >= \\ \hline
<ExprArith-p0-j> & epsilon + -  & <= <> or ) do then done else and by end endif ; to < = > >= \\ \hline
<ExprArith-p1-i> & [VarName] ( - [Number]  & <= <> or ) * + do then - done / else and by end endif ; to < = > >= \\ \hline
<ExprArith-p1-j> & epsilon * /  & <= <> or ) + do then - done else and by end endif ; to < = > >= 
\end{longtable}


    }
  %}%


\section{Action table}

To keep the table below concise, match and accept does not appear in the table below. Indeed, when the top of stack
contains a terminal, either the terminal is the same that the one on input and a match is performed (or an accept if the
terminal is ``\$''; either an error if the two terminals does not match.
\todo{A-t-on explique la semantique de chaque action?}

{\small\ttfamily
  \begin{tabular}{r|c@{ }c@{ }c@{ }c@{ }c@{ }c@{ }c@{ }c@{ }c@{ }c@{ }c@{ }c@{ }c@{ }c@{ }c@{ }c@{ }c@{ }c@{ }c@{ }c@{ }c@{ }c@{ }c@{ }c@{ }c@{ }c@{ }c@{ }}
 & ( & ) & * & + & - & / & ; & < & = & > & := & <= & <> & >= & by & do & if & or & to & and & end & for & not & done & else & from & rand \\\hline
<If> &   &   &   &   &   &   &   &   &   &   &   &   &   &   &   &   & 51 &   &   &   &   &   &   &   &   &   &   \\\hline
<For> &   &   &   &   &   &   &   &   &   &   &   &   &   &   &   &   &   &   &   &   &   & 54 &   &   &   &   &   \\\hline
<Atom> & 25 & 24 & 24 & 24 & 26 & 24 & 24 & 24 & 24 & 24 &   & 24 & 24 & 24 & 24 & 24 &   & 24 & 24 & 24 & 24 &   &   & 24 & 24 &   &   \\\hline
<Call> &   &   &   &   &   &   &   &   &   &   &   &   &   &   &   &   &   &   &   &   &   &   &   &   &   &   &   \\\hline
<Code> &   &   &   &   &   &   &   &   &   &   &   &   &   &   &   &   & 2 &   &   &   & 2 & 2 &   & 2 & 2 &   & 2 \\\hline
<Comp> & 43 &   &   &   & 43 &   &   & 42 & 38 & 40 &   & 41 & 43 & 39 &   &   &   &   &   &   &   &   &   &   &   &   &   \\\hline
<Rand> &   &   &   &   &   &   &   &   &   &   &   &   &   &   &   &   &   &   &   &   &   &   &   &   &   &   & 47 \\\hline
<Read> &   &   &   &   &   &   &   &   &   &   &   &   &   &   &   &   &   &   &   &   &   &   &   &   &   &   &   \\\hline
<Op-p0> & 28 &   &   & 27 & 28 &   &   &   &   &   &   &   &   &   &   &   &   &   &   &   &   &   &   &   &   &   &   \\\hline
<Op-p1> & 30 &   & 29 &   & 30 & 30 &   &   &   &   &   &   &   &   &   &   &   &   &   &   &   &   &   &   &   &   &   \\\hline
<Print> &   &   &   &   &   &   &   &   &   &   &   &   &   &   &   &   &   &   &   &   &   &   &   &   &   &   &   \\\hline
<While> &   &   &   &   &   &   &   &   &   &   &   &   &   &   &   &   &   &   &   &   &   &   &   &   &   &   &   \\\hline
<Assign> &   &   &   &   &   &   &   &   &   &   &   &   &   &   &   &   &   &   &   &   &   &   &   &   &   &   &   \\\hline
<Define> &   &   &   &   &   &   &   &   &   &   &   &   &   &   &   &   &   &   &   &   &   &   &   &   &   &   &   \\\hline
<Import> &   &   &   &   &   &   &   &   &   &   &   &   &   &   &   &   &   &   &   &   &   &   &   &   &   &   &   \\\hline
<Return> &   &   &   &   &   &   &   &   &   &   &   &   &   &   &   &   &   &   &   &   &   &   &   &   &   &   &   \\\hline
<Cond-p0> & 61 &   &   &   & 61 &   &   &   &   &   &   &   &   &   &   &   &   &   &   &   &   &   & 61 &   &   &   &   \\\hline
<Cond-p1> & 63 &   &   &   & 63 &   &   &   &   &   &   &   &   &   &   &   &   &   &   &   &   &   & 63 &   &   &   &   \\\hline
<Cond-p2> & 36 &   &   &   & 36 &   &   &   &   &   &   &   &   &   &   & 36 &   & 36 &   & 36 &   &   & 35 &   &   &   &   \\\hline
<If-Tail> &   &   &   &   &   &   & 52 &   &   &   &   &   &   &   &   &   &   &   &   &   & 52 &   &   & 52 & 53 &   &   \\\hline
<Program> &   &   &   &   &   &   &   &   &   &   &   &   &   &   &   &   &   &   &   &   &   &   &   &   &   &   &   \\\hline
<For-Tail> &   &   &   &   &   &   &   &   &   &   &   &   &   &   & 56 &   &   &   & 55 &   &   &   &   &   &   &   &   \\\hline
<InstList> &   &   &   &   &   &   &   &   &   &   &   &   &   &   &   &   & 48 &   &   &   &   & 48 &   &   &   &   & 48 \\\hline
<Cond-p0-i> & 32 &   &   &   & 32 &   &   &   &   &   &   &   &   &   &   & 32 &   & 32 &   &   &   &   & 32 &   &   &   &   \\\hline
<Cond-p0-j> &   &   &   &   &   &   &   &   &   &   &   &   &   &   &   & 62 &   & 31 &   &   &   &   &   &   &   &   &   \\\hline
<Cond-p1-i> & 34 &   &   &   & 34 &   &   &   &   &   &   &   &   &   &   & 34 &   & 34 &   & 34 &   &   & 34 &   &   &   &   \\\hline
<Cond-p1-j> &   &   &   &   &   &   &   &   &   &   &   &   &   &   &   & 64 &   & 64 &   & 33 &   &   &   &   &   &   &   \\\hline
<SimpleCond> & 37 &   &   &   & 37 &   &   &   &   &   &   &   &   &   &   &   &   &   &   &   &   &   &   &   &   &   &   \\\hline
<Instruction> &   &   &   &   &   &   & 13 &   &   &   &   &   &   &   &   &   & 4 &   &   &   & 13 & 6 &   & 13 & 13 &   & 9 \\\hline
<ExprArith-p0> & 57 &   &   &   & 57 &   &   &   &   &   &   &   &   &   &   &   &   &   &   &   &   &   &   &   &   &   &   \\\hline
<ExprArith-p1> & 59 &   &   &   & 59 &   &   &   &   &   &   &   &   &   &   &   &   &   &   &   &   &   &   &   &   &   &   \\\hline
<InstList-Tail> &   &   &   &   &   &   & 49 &   &   &   &   &   &   &   &   &   &   &   &   &   & 50 &   &   & 50 & 50 &   &   \\\hline
<ExprArith-p0-i> & 20 & 20 &   & 20 & 20 &   & 20 & 20 & 20 & 20 &   & 20 & 20 & 20 & 20 & 20 &   & 20 & 20 & 20 & 20 &   &   & 20 & 20 &   &   \\\hline
<ExprArith-p0-j> &   & 58 &   & 19 & 19 &   & 58 & 58 & 58 & 58 &   & 58 & 58 & 58 & 58 & 58 &   & 58 & 58 & 58 & 58 &   &   & 58 & 58 &   &   \\\hline
<ExprArith-p1-i> & 22 & 22 & 22 & 22 & 22 & 22 & 22 & 22 & 22 & 22 &   & 22 & 22 & 22 & 22 & 22 &   & 22 & 22 & 22 & 22 &   &   & 22 & 22 &   &   \\\hline
<ExprArith-p1-j> &   & 60 & 21 & 60 & 60 & 21 & 60 & 60 & 60 & 60 &   & 60 & 60 & 60 & 60 & 60 &   & 60 & 60 & 60 & 60 &   &   & 60 & 60 &   &   \\\hline
\end{tabular}


\begin{tabular}{r|c@{ }c@{ }c@{ }c@{ }c@{ }c@{ }c@{ }c@{ }c@{ }c@{ }c@{ }c@{ }c@{ }c@{ }c@{ }}
 & read & then & begin & endif & print & while & import & return & [Number] & function & [VarName] & [FuncName] & [ModuleName] & \$ & epsilon \\\hline
<If> &   &   &   &   &   &   &   &   &   &   &   &   &   &   &   \\\hline
<For> &   &   &   &   &   &   &   &   &   &   &   &   &   &   &   \\\hline
<Atom> &   & 24 &   & 24 &   &   &   &   & 24 &   & 23 &   &   &   & 24 \\\hline
<Call> &   &   &   &   &   &   &   &   &   &   &   & 17 &   &   &   \\\hline
<Code> & 2 &   &   & 2 & 2 & 2 & 2 & 2 &   & 2 & 2 & 2 &   &   & 2 \\\hline
<Comp> &   &   &   &   &   &   &   &   & 43 &   & 43 &   &   &   & 43 \\\hline
<Rand> &   &   &   &   &   &   &   &   &   &   &   &   &   &   &   \\\hline
<Read> & 46 &   &   &   &   &   &   &   &   &   &   &   &   &   &   \\\hline
<Op-p0> &   &   &   &   &   &   &   &   & 28 &   & 28 &   &   &   & 28 \\\hline
<Op-p1> &   &   &   &   &   &   &   &   & 30 &   & 30 &   &   &   & 30 \\\hline
<Print> &   &   &   &   & 45 &   &   &   &   &   &   &   &   &   &   \\\hline
<While> &   &   &   &   &   & 44 &   &   &   &   &   &   &   &   &   \\\hline
<Assign> &   &   &   &   &   &   &   &   &   &   & 18 &   &   &   &   \\\hline
<Define> &   &   &   &   &   &   &   &   &   & 14 &   &   &   &   &   \\\hline
<Import> &   &   &   &   &   &   & 16 &   &   &   &   &   &   &   &   \\\hline
<Return> &   &   &   &   &   &   &   & 15 &   &   &   &   &   &   &   \\\hline
<Cond-p0> &   &   &   &   &   &   &   &   & 61 &   & 61 &   &   &   &   \\\hline
<Cond-p1> &   &   &   &   &   &   &   &   & 63 &   & 63 &   &   &   &   \\\hline
<Cond-p2> &   & 36 &   &   &   &   &   &   & 36 &   & 36 &   &   &   & 36 \\\hline
<If-Tail> &   &   &   & 52 &   &   &   &   &   &   &   &   &   &   & 52 \\\hline
<Program> &   &   & 0 &   &   &   &   &   &   &   &   &   &   &   &   \\\hline
<For-Tail> &   &   &   &   &   &   &   &   &   &   &   &   &   &   &   \\\hline
<InstList> & 48 &   &   &   & 48 & 48 & 48 & 48 &   & 48 & 48 & 48 &   &   &   \\\hline
<Cond-p0-i> &   & 32 &   &   &   &   &   &   & 32 &   & 32 &   &   &   & 32 \\\hline
<Cond-p0-j> &   & 62 &   &   &   &   &   &   &   &   &   &   &   &   & 62 \\\hline
<Cond-p1-i> &   & 34 &   &   &   &   &   &   & 34 &   & 34 &   &   &   & 34 \\\hline
<Cond-p1-j> &   & 64 &   &   &   &   &   &   &   &   &   &   &   &   & 64 \\\hline
<SimpleCond> &   &   &   &   &   &   &   &   & 37 &   & 37 &   &   &   &   \\\hline
<Instruction> & 8 &   &   & 13 & 7 & 5 & 13 & 11 &   & 10 & 3 & 12 &   &   & 13 \\\hline
<ExprArith-p0> &   &   &   &   &   &   &   &   & 57 &   & 57 &   &   &   &   \\\hline
<ExprArith-p1> &   &   &   &   &   &   &   &   & 59 &   & 59 &   &   &   &   \\\hline
<InstList-Tail> &   &   &   & 50 &   &   &   &   &   &   &   &   &   &   & 50 \\\hline
<ExprArith-p0-i> &   & 20 &   & 20 &   &   &   &   & 20 &   & 20 &   &   &   & 20 \\\hline
<ExprArith-p0-j> &   & 58 &   & 58 &   &   &   &   &   &   &   &   &   &   & 58 \\\hline
<ExprArith-p1-i> &   & 22 &   & 22 &   &   &   &   & 22 &   & 22 &   &   &   & 22 \\\hline
<ExprArith-p1-j> &   & 60 &   & 60 &   &   &   &   &   &   &   &   &   &   & 60 \\\hline
\end{tabular}




}


\subsection{Imp grammar is now LL(1)}

One is able to prove that Imp grammar is now LL(1), only by checking that there is no conflict in the action table.
Two actions located in the same cell of the table would mean that, based on the same look-ahead token, 
there is two possible actions. When multiple actions can be performed without any additional information,
the parser becomes \textbf{non-deterministic}, and eventually make \textbf{wrong guesses} about which rule to apply.

Since each cell of our action table contains at most one possible action, there can be only one rule to apply to produce
any terminal from the input buffer. Since we use only \textbf{one symbol of look-ahead}, we can conclude
that the grammar is LL(1).
