\section{Implementation of the LL(1) parser}

\subsection{Theoretical considerations}

The objective is to implement a \textbf{recursive descent LL(1) parser} that tells whether a word (sequence of tokens)
is part of the language (Imp grammar) or not. This parser begins with the start variable and builds the
tree in a depth-first fashion using \textbf{production rules}. More specifically, it consists of a pushdown
automaton (PDA) that accepts the language by empty stack. Let's define this PDA formally:

\begin{equation}
  \begin{split}
    P_{Imp} = (\{q\}, T, V \cup T, \delta, q, \$, \phi)
   \end{split}
\end{equation}

\begin{itemize}
  \item Because $P_{Imp}$ accepts the grammar \textbf{by empty stack} and not by accepting state, it only requires the initial state to work. 
        Let's call this state $q$. The set of states is thus \{q\}.
  \item The alphabet of the input buffer is $T$. The set of terminals $T$ is the set of \textbf{lexical units} of Imp grammar. 
        Each terminal represents a type of token.
  \item The alphabet of the stack corresponds to $T \cup V$. Indeed, the stack is supposed to contain both lexical units and variables
        at some point, since variables are mandatory to use production rules.
  \item The transition function $\delta$ is such that the PDA can either \textbf{produce} or \textbf{match}. This will be explained
        in more details later on.
  \item The start symbol is $q$.
  \item We use \$ as the first symbol to push on the stack. When the stack becomes empty, the input buffer must have only one token
        remaining, which is \$. If it is not the case, the word is rejected.
  \item The set of accepting states is empty since we are not accepting words by accepting states, but by empty stack.
\end{itemize}

We implemented the parser using a stack instead of hard coding each rule as a unique Java method. Instead of making
\textbf{explicit recursive calls to functions}, each symbol that composes the right-hand part of the current rule is pushed
on the stack. This yields the same results than what we could get after implementing a

\subsection{Automating the process, but not too much}

We think that hard coding is bad. That is the reason why we did not described the grammar in the source code itself.
We encoded the Imp grammar (as it was presented in the project statement) in a dedicated \textit{more/grammars} folder and called \textit{imp.grammar}.
After that, we checked for useless rules and symbols (and found nothing) by hand but also programmatically, and finally dealed with
the operators priority and associativity by hand. The resulting grammar file, called \textit{imp\_prim.grammar} is used as input by our program.
We will refer to the later as the \textbf{unambiguous Imp grammar}.

We removed left-recursion, left-factored the grammar and finally computed first sets and follow sets of each grammar symbol by hand. 
Everything has been verified programmatically. When the parser is executed on an input Imp file, a lexer is instantiated to be able
to get the resulting list of tokens. Then this sequence of tokens is passed to the parser that first loads the unambiguous Imp grammar,
processes all the steps that we just enumerated \textbf{to make it LL(1)}, 
builds its action table from first sets and follow sets, and finally does the parsing itself.

\subsection{Grammar symbols and rules}

Because everything had to be implemented from scratch, we had to define what grammar symbols actually are. Because class \textit{Symbol}
was already present in our source code, and because it is (as explained earlier) fondamentally different from grammar symbols,
we created a new class \textit{GrammarSymbol}. A grammar symbol \textbf{can be either a variable or a terminal}. Everytime one does need
to know whether a grammar symbol is a terminal or not, he calls the method \textit{isTerminal}.

A \textit{Rule} is an object that consists of a left-hand variable represented by a \textit{GrammarSymbol} and a right-hand sequence
of symbols represented by a list of \textit{GrammarSymbol} objects. All rules are instantiated when loading the unambiguous grammar file.
For better productivity, we added special methods to display the rules, pretty print them or export them to LaTex.

\subsection{Action table}

Class \textit{ActionTable} has been implemented as a Map, where keys are pairs of \textit{GrammarSymbol} objects, and values are integers
that represent actions. Each key pair is composed of two parts: the symbol that is on top of the stack, and the next input terminal to read.
Integer values that are positive represent rule numbers, in particular the rule that must be used to produce. The
action table for the imp language is given below. Strictly negative integers
represent other types of actions: -1 stands for \textbf{match}, -2 stands for \textbf{accept} and -3 stands for \textbf{error}.

In source code, \textit{LL1Parser} class contains a stack, a reference to the action table, and the root of the tree. Also, it possesses helper
methods to export the parse to txt, javascript and latex.

We implemented our parser and parse tree as follows:
\begin{itemize}
  \item Push the \$ symbol on the stack and append the \$ symbol to the input sequence of tokens. Then, push the start symbol \varstyle{Program}
        on the stack.
  \item Look at the top of the stack, as well as to the next input token. Then combine the two symbols in a list and use it as a key to the
        action table. The action table returns an integer value that represents the action to apply. If the value is positive, then the parser 
        \textbf{produces}. If the value is -1, then the parser does a \textbf{match}. Finally if the value is -2, it means that both symbols
        are \$, and that the sequence of tokens has been \textbf{accepted by empty stack}. In all other cases, it produces a \textbf{syntax error}.
        Set the current node as the top of the stack.
  \item In case of a \textbf{produce}, use the integer value as the identifier of the rule to apply, then push all the symbols from the 
        right-hand side of the rule on the stack. Also, push a symbol only if it is not $\epsilon$. Create a new node containing the symbol
        and add it to the children list of the current node.
  \item In case of a \textbf{match}, pop the stack and look at the next token from the inputs.
  \item Repeat from second step, until an error occurs or the sequence of tokens gets accepted.
\end{itemize}

\section{Parse tree}

The parse tree is obviously built during parsing. Each \textit{Node} object is composed of a grammar symbol that represents it,
an identifier, a parent \textit{Node} object, and a list of children. Because children are \textbf{not created in the right order},
we implemented a \textit{NodeComparator} to be able to compare nodes by identifiers. Every time a new child is added to a parent node,
the latter sorts its children list using the \textit{NodeComparator}. Once the tree is built, we compile it to Javascript to be able
to display it in the browser.

\section{How to use the parser}

\begin{lstlisting}
Arguments :
	(1) --ru <grammar file> -o <grammar output file>
	(2) --ll <grammar file> -o <grammar output file>
	(3) --at <ll1 unambiguous grammar file> -o <latex output file>
	(4) <ll1 unambiguous grammar file> <code> [-o <parse tree output file>]
(1) remove useless rules and symbols
(2) left factorization and removing of left-recursion
(3) print action table
(4) save parse tree and output the rules used}
\end{lstlisting}

(1) You don't need to execute this command yourself. <grammar file> is the path to the initial grammar.
Our parser has been designed to work with any grammar. For this project, we removed useless rules and symbols
using the following arguments:
\begin{lstlisting}
--ru ../more/grammars/imp.grammar -o ../more/grammars/imp_ru.grammar
\end{lstlisting}

(2) You don't need to execute this command either. <grammar file> is the path to the unambigous grammar, which
means that it must contains no useless symbol nor ambiguities.
For this project, we transformed the grammar into a LL(1) grammar by using the following arguments:
\begin{lstlisting}
--ll ../more/grammars/imp_prim.grammar -o ../more/grammars/imp_ll.grammar
\end{lstlisting}

(3) You don't need to execute this command either. <grammar file> is the path to the unambigous LL(1) grammar.
We used it to compile our first sets, follow sets, and action table to LaTex.
\begin{lstlisting}
--at ../more/grammars/imp_ll.grammar -o action_table.tex
\end{lstlisting}

(4) Run the parser to get the output expected as in the project statement.
For example, to run the jar on one of out test files (you can obviously use yours), use the following arguments:
../more/grammars/imp_ll.grammar ../test/test-1.imp

\section{Visualize parse trees}

Because comparing rule identifiers from the command line to rules from a list can be tedious,
\textbf{we recommand you to use the parse tree}. We used Go.js for visualization.
To see the parse tree of the last Imp program that has been parsed, just open \textit{more/trees/tree.html}
in your browser (you will need an internet connection for that).
It is pretty straightforward to compare your input Imp program with the parse tree since all terminals
are aligned.
