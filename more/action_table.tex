\begin{tabular}{r|@{\hskip0.12em}c@{\hskip0.12em}c@{\hskip0.12em}c@{\hskip0.12em}c@{\hskip0.12em}c@{\hskip0.12em}c@{\hskip0.12em}c@{\hskip0.12em}c@{\hskip0.12em}c@{\hskip0.12em}c@{\hskip0.12em}c@{\hskip0.12em}c@{\hskip0.12em}c@{\hskip0.12em}c@{\hskip0.12em}c@{\hskip0.12em}c@{\hskip0.12em}c@{\hskip0.12em}c@{\hskip0.12em}c@{\hskip0.12em}c@{\hskip0.12em}c@{\hskip0.12em}c@{\hskip0.12em}c@{\hskip0.12em}c@{\hskip0.12em}c@{\hskip0.12em}c@{\hskip0.12em}c@{\hskip0.12em}c@{\hskip0.12em}c@{\hskip0.12em}c@{\hskip0.12em}c@{\hskip0.12em}c@{\hskip0.12em}c@{\hskip0.12em}c@{\hskip0.12em}c@{\hskip0.12em}c}
 & \verb=(= & \verb=)= & \verb=*= & \verb=+= & \verb=-= & \verb=/= & \verb=;= & \verb=<= & \verb=== & \verb=>= & \verb=:== & \verb=<== & \verb=<>= & \verb=>== & \verb=by= & \verb=do= & \verb=if= & \verb=or= & \verb=to= & \verb=and= & \verb=end= & \verb=for= & \verb=not= & \verb=done= & \verb=else= & \verb=from= & \verb=read= & \verb=then= & \verb=begin= & \verb=endif= & \verb=print= & \verb=while= & \verb=[Number]= & \verb=[VarName]= & \verb=$= & \verb=epsilon=\\
\verb=<If>= &   &   &   &   &   &   &   &   &   &   &   &   &   &   &   &   & 41 &   &   &   &   &   &   &   &   &   &   &   &   &   &   &   &   &   &   &   \\
\verb=<For>= &   &   &   &   &   &   &   &   &   &   &   &   &   &   &   &   &   &   &   &   &   & 44 &   &   &   &   &   &   &   &   &   &   &   &   &   &   \\
\verb=<Atom>= & 16 & 15 & 15 & 15 & 17 & 15 & 15 & 15 & 15 & 15 &   & 15 & 15 & 15 & 15 & 15 &   & 15 & 15 & 15 & 15 &   &   & 15 & 15 &   &   & 15 &   & 15 &   &   & 15 & 14 &   & 15 \\
\verb=<Code>= &   &   &   &   &   &   &   &   &   &   &   &   &   &   &   &   & 2 &   &   &   & 2 & 2 &   & 2 & 2 &   & 2 &   &   & 2 & 2 & 2 &   & 2 &   & 2 \\
\verb=<Comp>= & 34 &   &   &   & 34 &   &   & 33 & 29 & 31 &   & 32 & 34 & 30 &   &   &   &   &   &   &   &   &   &   &   &   &   &   &   &   &   &   & 34 & 34 &   & 34 \\
\verb=<Read>= &   &   &   &   &   &   &   &   &   &   &   &   &   &   &   &   &   &   &   &   &   &   &   &   &   &   & 37 &   &   &   &   &   &   &   &   &   \\
\verb=<Op-p0>= & 19 &   &   & 18 & 19 &   &   &   &   &   &   &   &   &   &   &   &   &   &   &   &   &   &   &   &   &   &   &   &   &   &   &   & 19 & 19 &   & 19 \\
\verb=<Op-p1>= & 21 &   & 20 &   & 21 & 21 &   &   &   &   &   &   &   &   &   &   &   &   &   &   &   &   &   &   &   &   &   &   &   &   &   &   & 21 & 21 &   & 21 \\
\verb=<Print>= &   &   &   &   &   &   &   &   &   &   &   &   &   &   &   &   &   &   &   &   &   &   &   &   &   &   &   &   &   &   & 36 &   &   &   &   &   \\
\verb=<While>= &   &   &   &   &   &   &   &   &   &   &   &   &   &   &   &   &   &   &   &   &   &   &   &   &   &   &   &   &   &   &   & 35 &   &   &   &   \\
\verb=<Assign>= &   &   &   &   &   &   &   &   &   &   &   &   &   &   &   &   &   &   &   &   &   &   &   &   &   &   &   &   &   &   &   &   &   & 9 &   &   \\
\verb=<Cond-p0>= & 51 &   &   &   & 51 &   &   &   &   &   &   &   &   &   &   &   &   &   &   &   &   &   & 51 &   &   &   &   &   &   &   &   &   & 51 & 51 &   &   \\
\verb=<Cond-p1>= & 53 &   &   &   & 53 &   &   &   &   &   &   &   &   &   &   &   &   &   &   &   &   &   & 53 &   &   &   &   &   &   &   &   &   & 53 & 53 &   &   \\
\verb=<Cond-p2>= & 27 &   &   &   & 27 &   &   &   &   &   &   &   &   &   &   & 27 &   & 27 &   & 27 &   &   & 26 &   &   &   &   & 27 &   &   &   &   & 27 & 27 &   & 27 \\
\verb=<If-Tail>= &   &   &   &   &   &   & 42 &   &   &   &   &   &   &   &   &   &   &   &   &   & 42 &   &   & 42 & 43 &   &   &   &   & 42 &   &   &   &   &   & 42 \\
\verb=<Program>= &   &   &   &   &   &   &   &   &   &   &   &   &   &   &   &   &   &   &   &   &   &   &   &   &   &   &   &   & 0 &   &   &   &   &   &   &   \\
\verb=<For-Tail>= &   &   &   &   &   &   &   &   &   &   &   &   &   &   & 46 &   &   &   & 45 &   &   &   &   &   &   &   &   &   &   &   &   &   &   &   &   &   \\
\verb=<InstList>= &   &   &   &   &   &   &   &   &   &   &   &   &   &   &   &   & 38 &   &   &   &   & 38 &   &   &   &   & 38 &   &   &   & 38 & 38 &   & 38 &   &   \\
\verb=<Cond-p0-i>= & 23 &   &   &   & 23 &   &   &   &   &   &   &   &   &   &   & 23 &   & 23 &   &   &   &   & 23 &   &   &   &   & 23 &   &   &   &   & 23 & 23 &   & 23 \\
\verb=<Cond-p0-j>= &   &   &   &   &   &   &   &   &   &   &   &   &   &   &   & 52 &   & 22 &   &   &   &   &   &   &   &   &   & 52 &   &   &   &   &   &   &   & 52 \\
\verb=<Cond-p1-i>= & 25 &   &   &   & 25 &   &   &   &   &   &   &   &   &   &   & 25 &   & 25 &   & 25 &   &   & 25 &   &   &   &   & 25 &   &   &   &   & 25 & 25 &   & 25 \\
\verb=<Cond-p1-j>= &   &   &   &   &   &   &   &   &   &   &   &   &   &   &   & 54 &   & 54 &   & 24 &   &   &   &   &   &   &   & 54 &   &   &   &   &   &   &   & 54 \\
\verb=<SimpleCond>= & 28 &   &   &   & 28 &   &   &   &   &   &   &   &   &   &   &   &   &   &   &   &   &   &   &   &   &   &   &   &   &   &   &   & 28 & 28 &   &   \\
\verb=<Instruction>= &   &   &   &   &   &   & 8 &   &   &   &   &   &   &   &   &   & 4 &   &   &   & 8 & 6 &   & 8 & 8 &   & 8 &   &   & 8 & 7 & 5 &   & 3 &   & 8 \\
\verb=<ExprArith-p0>= & 47 &   &   &   & 47 &   &   &   &   &   &   &   &   &   &   &   &   &   &   &   &   &   &   &   &   &   &   &   &   &   &   &   & 47 & 47 &   &   \\
\verb=<ExprArith-p1>= & 49 &   &   &   & 49 &   &   &   &   &   &   &   &   &   &   &   &   &   &   &   &   &   &   &   &   &   &   &   &   &   &   &   & 49 & 49 &   &   \\
\verb=<InstList-Tail>= &   &   &   &   &   &   & 39 &   &   &   &   &   &   &   &   &   &   &   &   &   & 40 &   &   & 40 & 40 &   &   &   &   & 40 &   &   &   &   &   & 40 \\
\verb=<ExprArith-p0-i>= & 11 & 11 &   & 11 & 11 &   & 11 & 11 & 11 & 11 &   & 11 & 11 & 11 & 11 & 11 &   & 11 & 11 & 11 & 11 &   &   & 11 & 11 &   &   & 11 &   & 11 &   &   & 11 & 11 &   & 11 \\
\verb=<ExprArith-p0-j>= &   & 48 &   & 10 & 10 &   & 48 & 48 & 48 & 48 &   & 48 & 48 & 48 & 48 & 48 &   & 48 & 48 & 48 & 48 &   &   & 48 & 48 &   &   & 48 &   & 48 &   &   &   &   &   & 48 \\
\verb=<ExprArith-p1-i>= & 13 & 13 & 13 & 13 & 13 & 13 & 13 & 13 & 13 & 13 &   & 13 & 13 & 13 & 13 & 13 &   & 13 & 13 & 13 & 13 &   &   & 13 & 13 &   &   & 13 &   & 13 &   &   & 13 & 13 &   & 13 \\
\verb=<ExprArith-p1-j>= &   & 50 & 12 & 50 & 50 & 12 & 50 & 50 & 50 & 50 &   & 50 & 50 & 50 & 50 & 50 &   & 50 & 50 & 50 & 50 &   &   & 50 & 50 &   &   & 50 &   & 50 &   &   &   &   &   & 50 \\
\hline
\verb=(= & -1 &   &   &   &   &   &   &   &   &   &   &   &   &   &   &   &   &   &   &   &   &   &   &   &   &   &   &   &   &   &   &   &   &   &   &  \\
\verb=)= &   & -1 &   &   &   &   &   &   &   &   &   &   &   &   &   &   &   &   &   &   &   &   &   &   &   &   &   &   &   &   &   &   &   &   &   &  \\
\verb=*= &   &   & -1 &   &   &   &   &   &   &   &   &   &   &   &   &   &   &   &   &   &   &   &   &   &   &   &   &   &   &   &   &   &   &   &   &  \\
\verb=+= &   &   &   & -1 &   &   &   &   &   &   &   &   &   &   &   &   &   &   &   &   &   &   &   &   &   &   &   &   &   &   &   &   &   &   &   &  \\
\verb=-= &   &   &   &   & -1 &   &   &   &   &   &   &   &   &   &   &   &   &   &   &   &   &   &   &   &   &   &   &   &   &   &   &   &   &   &   &  \\
\verb=/= &   &   &   &   &   & -1 &   &   &   &   &   &   &   &   &   &   &   &   &   &   &   &   &   &   &   &   &   &   &   &   &   &   &   &   &   &  \\
\verb=;= &   &   &   &   &   &   & -1 &   &   &   &   &   &   &   &   &   &   &   &   &   &   &   &   &   &   &   &   &   &   &   &   &   &   &   &   &  \\
\verb=<= &   &   &   &   &   &   &   & -1 &   &   &   &   &   &   &   &   &   &   &   &   &   &   &   &   &   &   &   &   &   &   &   &   &   &   &   &  \\
\verb=== &   &   &   &   &   &   &   &   & -1 &   &   &   &   &   &   &   &   &   &   &   &   &   &   &   &   &   &   &   &   &   &   &   &   &   &   &  \\
\verb=>= &   &   &   &   &   &   &   &   &   & -1 &   &   &   &   &   &   &   &   &   &   &   &   &   &   &   &   &   &   &   &   &   &   &   &   &   &  \\
\verb=:== &   &   &   &   &   &   &   &   &   &   & -1 &   &   &   &   &   &   &   &   &   &   &   &   &   &   &   &   &   &   &   &   &   &   &   &   &  \\
\verb=<== &   &   &   &   &   &   &   &   &   &   &   & -1 &   &   &   &   &   &   &   &   &   &   &   &   &   &   &   &   &   &   &   &   &   &   &   &  \\
\verb=<>= &   &   &   &   &   &   &   &   &   &   &   &   & -1 &   &   &   &   &   &   &   &   &   &   &   &   &   &   &   &   &   &   &   &   &   &   &  \\
\verb=>== &   &   &   &   &   &   &   &   &   &   &   &   &   & -1 &   &   &   &   &   &   &   &   &   &   &   &   &   &   &   &   &   &   &   &   &   &  \\
\verb=by= &   &   &   &   &   &   &   &   &   &   &   &   &   &   & -1 &   &   &   &   &   &   &   &   &   &   &   &   &   &   &   &   &   &   &   &   &  \\
\verb=do= &   &   &   &   &   &   &   &   &   &   &   &   &   &   &   & -1 &   &   &   &   &   &   &   &   &   &   &   &   &   &   &   &   &   &   &   &  \\
\verb=if= &   &   &   &   &   &   &   &   &   &   &   &   &   &   &   &   & -1 &   &   &   &   &   &   &   &   &   &   &   &   &   &   &   &   &   &   &  \\
\verb=or= &   &   &   &   &   &   &   &   &   &   &   &   &   &   &   &   &   & -1 &   &   &   &   &   &   &   &   &   &   &   &   &   &   &   &   &   &  \\
\verb=to= &   &   &   &   &   &   &   &   &   &   &   &   &   &   &   &   &   &   & -1 &   &   &   &   &   &   &   &   &   &   &   &   &   &   &   &   &  \\
\verb=and= &   &   &   &   &   &   &   &   &   &   &   &   &   &   &   &   &   &   &   & -1 &   &   &   &   &   &   &   &   &   &   &   &   &   &   &   &  \\
\verb=end= &   &   &   &   &   &   &   &   &   &   &   &   &   &   &   &   &   &   &   &   & -1 &   &   &   &   &   &   &   &   &   &   &   &   &   &   &  \\
\verb=for= &   &   &   &   &   &   &   &   &   &   &   &   &   &   &   &   &   &   &   &   &   & -1 &   &   &   &   &   &   &   &   &   &   &   &   &   &  \\
\verb=not= &   &   &   &   &   &   &   &   &   &   &   &   &   &   &   &   &   &   &   &   &   &   & -1 &   &   &   &   &   &   &   &   &   &   &   &   &  \\
\verb=done= &   &   &   &   &   &   &   &   &   &   &   &   &   &   &   &   &   &   &   &   &   &   &   & -1 &   &   &   &   &   &   &   &   &   &   &   &  \\
\verb=else= &   &   &   &   &   &   &   &   &   &   &   &   &   &   &   &   &   &   &   &   &   &   &   &   & -1 &   &   &   &   &   &   &   &   &   &   &  \\
\verb=from= &   &   &   &   &   &   &   &   &   &   &   &   &   &   &   &   &   &   &   &   &   &   &   &   &   & -1 &   &   &   &   &   &   &   &   &   &  \\
\verb=read= &   &   &   &   &   &   &   &   &   &   &   &   &   &   &   &   &   &   &   &   &   &   &   &   &   &   & -1 &   &   &   &   &   &   &   &   &  \\
\verb=then= &   &   &   &   &   &   &   &   &   &   &   &   &   &   &   &   &   &   &   &   &   &   &   &   &   &   &   & -1 &   &   &   &   &   &   &   &  \\
\verb=begin= &   &   &   &   &   &   &   &   &   &   &   &   &   &   &   &   &   &   &   &   &   &   &   &   &   &   &   &   & -1 &   &   &   &   &   &   &  \\
\verb=endif= &   &   &   &   &   &   &   &   &   &   &   &   &   &   &   &   &   &   &   &   &   &   &   &   &   &   &   &   &   & -1 &   &   &   &   &   &  \\
\verb=print= &   &   &   &   &   &   &   &   &   &   &   &   &   &   &   &   &   &   &   &   &   &   &   &   &   &   &   &   &   &   & -1 &   &   &   &   &  \\
\verb=while= &   &   &   &   &   &   &   &   &   &   &   &   &   &   &   &   &   &   &   &   &   &   &   &   &   &   &   &   &   &   &   & -1 &   &   &   &  \\
\verb=[Number]= &   &   &   &   &   &   &   &   &   &   &   &   &   &   &   &   &   &   &   &   &   &   &   &   &   &   &   &   &   &   &   &   & -1 &   &   &  \\
\verb=[VarName]= &   &   &   &   &   &   &   &   &   &   &   &   &   &   &   &   &   &   &   &   &   &   &   &   &   &   &   &   &   &   &   &   &   & -1 &   &  \\
\verb=$= &   &   &   &   &   &   &   &   &   &   &   &   &   &   &   &   &   &   &   &   &   &   &   &   &   &   &   &   &   &   &   &   &   &   & -2 &  \\
\end{tabular}
